% arara: xelatex: { synctex: true, action: nonstopmode, options: "-halt-on-error" }
% arara: clean: { files: [ 2017-02-27.aux, 2017-02-27.bbl, 2017-02-27.bcf, 2017-02-27.blg, 2017-02-27.log, 2017-02-27.out, 2017-02-27.run.xml ]}
\documentclass[11pt]{report}
\usepackage{color}
\usepackage{hyperref}
\hypersetup{colorlinks = true}
\pagenumbering{roman}
\topmargin=0.0in
\oddsidemargin=0.0in
\evensidemargin=0.0in
\textwidth=6.5in
\marginparwidth=0.5in
\headheight=0pt
\headsep=0pt
\textheight=9.0in
\begin{document}

\centerline{{\LARGE \bf Atos Mentor Matching}}

\centerline{\large \bf Biweekly Report 7, 27\textsuperscript{th} of February 2017}
\centerline{Christopher Lau $\bullet$ Harry Chen $\bullet$ Matthew Bell }
\noindent
\line(1,0){470}\\


\noindent {\Large \bf Overview}
\smallskip

\noindent
Over the last two weeks, we have been working hard on both the web server and
the webapp to meet our project targets by the end of this term. The server now
successfully uses the OAuth2.0 protocol for user login, and the webapp is being
programmed to use these login and signup services. We are also working towards
setting up some mentorship-related API endpoints, such as the programme
creation/deletion endpoints and the cohort signup endpoints.

\bigskip
\noindent{\Large \bf Meetings To Date}
\smallskip

\noindent
We had no meetings this past fortnight since the first week was a reading week
and the second week was a scenario week where we worked on a different project
for the week.

\bigskip
\noindent{\Large \bf Completed Tasks}
\smallskip

\noindent
Since the previous report these are the tasks that have been completed to date:
\begin{itemize}
    \item Data service for transporting form data across templates finished.
    \item Add functionalities to all form pages.
    \item Hosted web app on ionic view for testing on mobile platform.
    \item Completed front-end templates for individual mentorship program description pages.
\end{itemize}

\bigskip
\noindent{\Large \bf Chris' Progerss}
\smallskip

\noindent
For the past two weeks, I continued working on designing front-end templates for
the webapp and also added interactive features using AngularJS to improve user
interface. I have also been designing controllers to send and receive data from
the REST API\@. I have also setup and hosted the webapp in Ionic View for testing
and displaying the design of the webapp in a mobile platform. The next steps are
to continue working on and improving the design of the templates for the
remaining pages of the webapp and integrating them with the REST API once it is
ready.

\bigskip
\noindent{\Large \bf Harry's Progress}
\smallskip

\noindent
These two weeks I worked on setting up data service for transporting user
defined data across templates, also solved cache problem from routing side. Now
the registration part of our app has completed functionality. All the data
(program name, interests, preferences) could be stored across the live time of
app. Once authentication finished we could test directly.

\bigskip
\noindent{\Large \bf Matt's Progress}
\smallskip

\noindent I have managed to combine Django’s User model with our custom
UserProfile model, meaning that we have a fully functional user model for the
server. Thanks to this, I was then able to set up an authentication scheme on
the server for applications to connect and login. We’re using the OAuth2.0
specification, which means that the web app will be registered as an application
on the server before use. This means that any unauthorised applications cannot
use our API\@. The next step is to set up user signup and profile editing,
followed by mentorship endpoints.

\bigskip
\noindent{\Large \bf Plan For The Next Two Weeks}
\smallskip

\noindent Continue working on designing front-end templates which are to be
integrated with REST calls to obtain user profile data.

\hrule

\subsection*{Keeping Up With The Project}

To complement these biweekly reports, you can also view these pages:\\

\noindent
Project site: http://students.cs.ucl.ac.uk/2016/group19/\\

\noindent
Code repository: https://github.com/mbellgb/syseng19-code\\

\noindent
Documentation: https://github.com/mbellgb/syseng19-docs

\end{document}
% vim: textwidth=80
