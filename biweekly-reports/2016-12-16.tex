% arara: xelatex: { synctex: true, action: nonstopmode, options: "-halt-on-error" }
% arara: clean: { files: [ 2016-12-16.aux, 2016-12-16.bbl, 2016-12-16.bcf, 2016-12-16.blg, 2016-12-16.log, 2016-12-16.out, 2016-12-16.run.xml ]}
\documentclass[11pt]{report}
\usepackage{color}
\usepackage{hyperref}
\hypersetup{colorlinks = true}
\pagenumbering{roman}
\topmargin=0.0in
\oddsidemargin=0.0in
\evensidemargin=0.0in
\textwidth=6.5in
\marginparwidth=0.5in
\headheight=0pt
\headsep=0pt
\textheight=9.0in
\begin{document}

\centerline{{\LARGE \bf Atos Mentor Matching}}

\centerline{\large \bf Biweekly Report 4, 16\textsuperscript{th} of December 2016}
\centerline{Christopher Lau $\bullet$ Harry Chen $\bullet$ Matthew Bell }
\noindent
\line(1,0){470}\\


\noindent {\Large \bf Overview}
\smallskip

\noindent
Over the last two weeks, the team concluded and finalised key aspects of the
development of this application such as requirements, technologies used, first
prototype etc. As of today, we currently have a completed and finalised first
prototype of the final application which consists of the overall user interface
of the final system. We have also designed the matching algorithm which is an
essential component in our system. Pseudocode for the algorithm was also
designed so that the algorithm could be implemented in the system during term 2.
An experiment log was also created which illustrates all the experiments we have
conducted so far and has been uploaded to the project website.

\bigskip
\noindent{\Large \bf Meetings To Date}
\smallskip

\noindent
We did not have a meeting with our client or supervisor (because of scheduling
conflicts). In last week’s team meeting we discussed the design prototype and
then modified it based on our comments. Also, we discussed the content we put on
website with our TA and amongst ourselves.

\bigskip
\noindent{\Large \bf Completed Tasks}
\smallskip

\noindent
Since the previous report these are the tasks that have been completed to date:
\begin{itemize}
    \item Finish app design prototype
    \item Create use cases
    \item Complete experiment log
    \item Add final elements to website
\end{itemize}

\bigskip
\noindent{\Large \bf Chris' Progerss}
\smallskip

\noindent
Worked on finishing the use case diagrams and also experiment log with Harry.
Also continued to add content to the project website and structuring the website
in an orderly and neat fashion. Also continued to manage communication with our
client and updating her on our progress.

\bigskip
\noindent{\Large \bf Harry's Progress}
\smallskip

\noindent
In last two weeks we received information (this including tag information and
description of each mentor program held by ATOS) from our client. Then I worked
out a persona and use cases of our app based on those information. Also I
created experiment log then Chris completed it with more details and put it on
website. In the end I modified app prototype with client’s information.

\bigskip
\noindent{\Large \bf Matt's Progress}
\smallskip

\noindent
I set up the directory structure on our code repository so that we’re ready to
start building the system as soon as possible when we return in January. I’ve
also been researching python libraries to use in the backend system.

\bigskip
\noindent{\Large \bf Plan For The Next Two Weeks}
\smallskip

\noindent
In next two weeks we will learn techniques that we’ll use for implementing our
project, including AngularJS and Django. And we’ll start working on the project
video.

\hrule

\subsection*{Keeping Up With The Project}

To complement these biweekly reports, you can also view these pages:\\

\noindent
Project site: http://students.cs.ucl.ac.uk/2016/group19/\\

\noindent
Code repository: https://github.com/mbellgb/syseng19-code\\

\noindent
Documentation: https://github.com/mbellgb/syseng19-docs

\end{document}
% vim: textwidth=80
