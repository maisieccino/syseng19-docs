% arara: xelatex: { synctex: true, action: nonstopmode, options: "-halt-on-error" }
% arara: clean: { files: [ 2017-02-10.aux, 2017-02-10.bbl, 2017-02-10.bcf, 2017-02-10.blg, 2017-02-10.log, 2017-02-10.out, 2017-02-10.run.xml ]}
\documentclass[11pt]{report}
\usepackage{color}
\usepackage{hyperref}
\hypersetup{colorlinks = true}
\pagenumbering{roman}
\topmargin=0.0in
\oddsidemargin=0.0in
\evensidemargin=0.0in
\textwidth=6.5in
\marginparwidth=0.5in
\headheight=0pt
\headsep=0pt
\textheight=9.0in
\begin{document}

\centerline{{\LARGE \bf Atos Mentor Matching}}

\centerline{\large \bf Biweekly Report 6, 10\textsuperscript{th} of February 2017}
\centerline{Christopher Lau $\bullet$ Harry Chen $\bullet$ Matthew Bell }
\noindent
\line(1,0){470}\\


\noindent {\Large \bf Overview}
\smallskip

\noindent
The primary focus of the past two weeks was to begin development of the API and
webapp, as well as setting up development tools to aid integration and testing
as development progresses. We also spent some time to touch base with our
client.

\bigskip
\noindent{\Large \bf Meetings To Date}
\smallskip

\noindent
\begin{itemize}
    \item \textbf{Supervisor Meeting (2017/02/03)}: The meeting involved
        updating our supervisor regarding the progress that we have made so far
        in terms of development and receiving feedback on things we could
        improve on. Our supervisor also gave us suggestions on how we could
        prepare for our final presentation which was in early April. We agreed
        that we could prepare short presentation slides prior to every
        supervisor meeting illustrating our progress so that our supervisor
        could give feedback that would contribute to our final presentation.
    \item \textbf{Client Meeting (2017/02/08)}: This was a brief meeting to
        update our client regarding our progress and also discuss about the
        project website and prototype design. Overall, the client was satisfied
        with the documentation that the project website contained and was happy
        with our prototype design. Our client also mentioned that she will be
        sending us the dates for which she will be in London so that we could
        arrange a face to face meeting with her later towards the end of the
        month.
\end{itemize}

\bigskip
\noindent{\Large \bf Completed Tasks}
\smallskip

\noindent
Since the previous report these are the tasks that have been completed to date:
\begin{itemize}
    \item Webapp routing added.
    \item Research of token based authentication.
    \item Front-end templates for the home page and user profile page which are to be integrated with user authentication when it is complete.
    \item Set up test server. (https://api.dev.mbell.me)
    \item First version of API documentation created.
    \item Set up continuous integration and unit testing harness.
    \item Server-side data models created and initial database migrations added.
\end{itemize}

\bigskip
\noindent{\Large \bf Chris' Progerss}
\smallskip

\noindent
For the past two weeks, I was working on creating templates and wireframes for
the homepage and user profile page using ionic UI components and AngularJS\@.
These templates will be integrated with the backend once the user authentication
portion of REST API is complete. I was also working on improving the overall UI
design of the web app such as cleaning up the appearance of the side menu and
fixed some of the routing issues that the web app contained. Besides that, I
continued to manage communication with the client by arranging meetings with her
and keeping her up to date with our progress.

\bigskip
\noindent{\Large \bf Harry's Progress}
\smallskip

\noindent
These two weeks I'm working on the user authentication. After talking with Matt
we decided to use token instead of session and cookies. Then I took a research
of the local storage in ionic app, import and use the module ngstorage. The
controllers and services for authentication are almost done but there are still
some problems need to be solved. By the way I solved all the routing inside the
app, also set up the default page to login page.

\bigskip
\noindent{\Large \bf Matt's Progress}
\smallskip

\noindent
These past two weeks, I have been incredibly busy completing some further devops
tasks, as well as working on the API to set up our ``CRUD'' (Create, Read, Update,
Destroy) and user authentication endpoints. Firstly, I have acquired a
DigitalOcean VPS and configured it to run Django safely and securely. I then
created a remote Git repository on the server as well as some Git-hooks which
deploy any code pushed to that remote. Next, I set up HTTPS on the server using
Let’s Encrypt, meaning that any traffic to/from the server is encrypted from
third parties. I then created some basic unit tests for the code so far, added a
config file, and set up Travis CI so that our code is staged, tested and then
deployed to our test server automatically, thus enabling us to stick to the
Agile methodology a lot closer than otherwise, as well as ensuring code is
always well-tested. We’re using Test-Driven Development (TDD) for the server
code, meaning that tests are written before we write the actual application
code. Finally, in terms of the code itself, I have successfully transferred
models from the UML diagrams made earlier into Django data models, and have
successfully migrated these into staging databases both locally and on the test
server. The next steps are to set up user authentication so that we can start
integrating the web app with the API\@.

\bigskip
\noindent{\Large \bf Plan For The Next Two Weeks}
\smallskip

\noindent
\begin{itemize}
    \item Finish the user authentication in front-end side.
    \item Continue working to improve the user interface design and integrate
        the front-end with the backend REST API\@.
\end{itemize}

\hrule

\subsection*{Keeping Up With The Project}

To complement these biweekly reports, you can also view these pages:\\

\noindent
Project site: http://students.cs.ucl.ac.uk/2016/group19/\\

\noindent
Code repository: https://github.com/mbellgb/syseng19-code\\

\noindent
Documentation: https://github.com/mbellgb/syseng19-docs

\end{document}
% vim: textwidth=80
