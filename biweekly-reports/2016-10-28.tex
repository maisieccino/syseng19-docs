\documentclass[11pt]{report}
\usepackage{color}
\usepackage{hyperref}
\hypersetup{
  colorlinks = true
}
\pagenumbering{roman}
\topmargin=\item 0in
\oddsidemargin=\item 0in
\evensidemargin=\item 0in
\textwidth=\item 5in
\marginparwidth=\item 5in
\headheight=0pt
\headsep=0pt
\textheight=\item 0in
\begin{document}

\centerline{{\LARGE \bf Atos Mentor Matching}}

\centerline{ \large \bf Biweekly Report \item  28\textsuperscript{th} of October 2016}
\centerline{ Christopher Lau $\bullet$ Harry Chen $\bullet$ Matthew Bell }
\noindent
\line(\item 0){470}\\


\noindent {\Large \bf Overview}
\smallskip

\noindent
The last two weeks involved taking what was learned from meeting our client and forming a set of requirements. We also set up our development methodology and early designs of both the user interface and the system architecture.

\bigskip
\noindent{\Large \bf Meetings To Date}
\smallskip

\begin{description}
  \item[Supervisor Meeting \item  24\textsuperscript{th} of October 2016] --- During the first meeting with our supervisor, we introduced each other and gave a quick brief of our project. She also gave some advice on how to manage the project and how to write the project outline.
  \item[Team Meeting \item  26\textsuperscript{th} of October 2016] --- We discussed how to manage the project website; at first Harry built a website by using Bootstrap and Less. But later Matt suggested using Jekyll to build a static webise which is much easier to maintain. We also discussed the technology stack we going to use for the app and shared system designs.
  \item[Client Meeting \item  27\textsuperscript{th} of October 2016] --- This was a quick web call to finalize requirements. The discussion was more focused on the process of matching mentors with mentees and also selected fields which were to be included in user profiles for the app. Our client and us also raised any outstanding confusion regarding the general structure of the app (eg. The different mentoring schemes should have different forms).
\end{description}

\bigskip
\noindent{\Large \bf Completed Tasks}
\smallskip

\noindent
Since the previous report these are the tasks that have been completed to date:
\begin{itemize}
    \item Create the website for our project (http://students.cs.ucl.ac.uk/2016/group19/)
    \item Research existing apps, e.g. Momo, Mentor Match
    \item Initial sketches of user interface
    \item Formalise project requirements
    \item Arrange meeting with another client (Mike) for technical side requirements
    \item Create entity-relationship diagram
\end{itemize}

\bigskip
\noindent{\Large \bf Chris' Progress}
\smallskip

\noindent
For this two weeks I managed most of the communication with the client and arrange meetings. I also made some notes on the recent meeting with our Client. I also managed pull a working version of the project site and will be working on assisting Matt to add some content.\\

\bigskip
\noindent{\Large \bf Harry's Progress}
\smallskip

\noindent
Figure out the first verison of team website(by Bootstrap) and continue working on user interface design.Did the bi-weekly report also take some research of existing systems.\\

\pagebreak
\bigskip
\noindent{\Large \bf Matt's Progress}
\smallskip

\noindent
At the start of the fortnight, I formalised the requirements for the project and sent them off to the client. Then, I set up a basic static site using jekyll and have started adding content to it. The nature of the site means that we can very quickly add new content to it without having to touch any HTML code. Finally, I have been sketching some system architecture designs (e.g., user authentication API, entity-relation diagram), and have been researching the software stack that we should use.\\

\bigskip
\noindent{\Large \bf Problems To Resolve}
\smallskip

\noindent

\begin{itemize}
    \item Problems to be solved when making project website - e.g., acquiring project server from department.
    \item Research the matching examples from client and make a framework.
\end{itemize}

\bigskip
\noindent{\Large \bf Plan For The Next Two Weeks}
\smallskip

\noindent
Over the next two weeks we will be focusing on the following activities:

\begin{itemize}
 \item Take the mentor matching example from client
 \item Meet Mike (Chief Technology Officer) and get requirements from tech side
 \item Continue working on requirements and research.
 \item Existing / example code investigations should begin.
 \item Create project website for client, TAs and supervisor.
 \item Present Initial analysis of literature findings to client.
 \item with the research identifying strategies, ideas, tools, etc. Compare Software Engineering Methods
\end{itemize}

\end{document}
