% arara: xelatex: { synctex: true, action: nonstopmode, options: "-halt-on-error" }
% arara: clean: { files: [ 2017-03-24.aux, 2017-03-24.bbl, 2017-03-24.bcf, 2017-03-24.blg, 2017-03-24.log, 2017-03-24.out, 2017-03-24.run.xml ]}
\documentclass[11pt]{report}
\usepackage{color}
\usepackage{hyperref}
\hypersetup{colorlinks = true}
\pagenumbering{roman}
\topmargin=0.0in
\oddsidemargin=0.0in
\evensidemargin=0.0in
\textwidth=6.5in
\marginparwidth=0.5in
\headheight=0pt
\headsep=0pt
\textheight=9.0in
\begin{document}

\centerline{{\LARGE \bf Atos Mentor Matching}}

\centerline{\large \bf Biweekly Report 9, 24\textsuperscript{th} of March 2017}
\centerline{Christopher Lau $\bullet$ Harry Chen $\bullet$ Matthew Bell }
\noindent
\line(1,0){470}\\


\noindent {\Large \bf Overview}
\smallskip

\noindent
Over the last two weeks, we have made significant progress on both the frontend
and backend. The backend API has been updated to allow easier tag management, as
well as operations for cohorts (create, read, update, destroy), and an endpoint
that allows users to register for a programme’s cohort.

The webapp now performs checks whether the currently logged in user is an admin.
Admin features were also implemented for example creation of mentor programs and
also cohort creation of each program. Admins are now able to manage their
desired programs and cohorts by controlling the cohort size, and also start
date,end date and match date of the cohort. In addition, users could now modify
their profile, include uploading pictures and change their bio.

\bigskip
\noindent{\Large \bf Meetings To Date}
\smallskip

\noindent
\begin{itemize}
    \item \textbf{Team Meeting (2017/03/22)}: During this meeting, we worked on
        fixing API integration issues, as well as receive feedback on our
        website from Yun Fu.
\end{itemize}

\bigskip
\noindent{\Large \bf Completed Tasks}
\smallskip

\noindent
Since the previous report these are the tasks that have been completed to date:
\begin{itemize}
    \item Implemented the Cohort model and created the /cohort/ and
        /cohort/{cohortId}/ API endpoints
    \item Fixed some security issues, including object IDs being modifiable and
        cohort objects not patching properly
    \item Allow cohorts to be set with defaultCohortSize
    \item Allow tags to be created and fetched using strings
    \item Add /cohort/{cohortId}/register endpoint, allowing users to register
        for a programme cohort.
    \item Implemented user settings so that users can edit their profile.
    \item Implemented program creation for admins.
    \item Implemented program editing and cohort management for admins.
    \item Implemented image uploading in the front end side.
\end{itemize}

\bigskip
\noindent{\Large \bf Chris' Progerss}
\smallskip

\noindent
 have been continuing my work on the user interface and design of the webapp.
 For these two weeks, I have also implemented the user settings feature which
 allows users to modify their profile. Besides that, I have also implemented the
 checks for admin privileges, which means that admin privileges are granted only
 if the user is logged in as an admin. I have also worked on the UI of the
 cohort management page by listing out the cohort in a neat manner which is easy
 to read and manage. Finally, I have also been working on documentation by
 updating the project website.

\bigskip
\noindent{\Large \bf Harry's Progress}
\smallskip

\noindent
Last two weeks I worked on implementing functionalities of front-end side. After
Matt updated backend API, I achieved operations for cohorts and programs
(creating, modify, delete by admin). And I also take a research of cordova
camera API, achieve the image uploading function inside our app.

\bigskip
\noindent{\Large \bf Matt's Progress}
\smallskip

\noindent
I have been working hard on the API, making sure that it has a good coverage of
tests, making sure that features are implemented according to the API spec and
web standards, as well as ensuring that the test server does not crash when
updating. Cohort endpoints are now implemented, as well as the cohort
registration endpoint, meaning that users can now enrol themselves on mentorship
programmes.

\bigskip
\noindent{\Large \bf Plan For The Next Two Weeks}
\smallskip

\noindent
\begin{itemize}
    \item Implement /programme/{id}/cohorts/active endpoint, that returns the
        current “active” cohort for a programme.
    \item Implement basic matching algorithm, that is run on a cohort.
    \item Implement match serializer and endpoint.
    \item Design constraint system for matching and registering for programmes.
    \item Work on cohort registration on client side
\end{itemize}

\hrule

\subsection*{Keeping Up With The Project}

To complement these biweekly reports, you can also view these pages:\\

\noindent
Project site: http://students.cs.ucl.ac.uk/2016/group19/\\

\noindent
Code repository: https://github.com/mbellgb/syseng19-code\\

\noindent
Documentation: https://github.com/mbellgb/syseng19-docs

\end{document}
% vim: textwidth=80
