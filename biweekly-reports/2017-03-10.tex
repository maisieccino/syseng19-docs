% arara: xelatex: { synctex: true, action: nonstopmode, options: "-halt-on-error" }
% arara: clean: { files: [ 2017-03-10.aux, 2017-03-10.bbl, 2017-03-10.bcf, 2017-03-10.blg, 2017-03-10.log, 2017-03-10.out, 2017-03-10.run.xml ]}
\documentclass[11pt]{report}
\usepackage{color}
\usepackage{hyperref}
\hypersetup{colorlinks = true}
\pagenumbering{roman}
\topmargin=0.0in
\oddsidemargin=0.0in
\evensidemargin=0.0in
\textwidth=6.5in
\marginparwidth=0.5in
\headheight=0pt
\headsep=0pt
\textheight=9.0in
\begin{document}

\centerline{{\LARGE \bf Atos Mentor Matching}}

\centerline{\large \bf Biweekly Report 8, 10\textsuperscript{th} of March 2017}
\centerline{Christopher Lau $\bullet$ Harry Chen $\bullet$ Matthew Bell }
\noindent
\line(1,0){470}\\


\noindent {\Large \bf Overview}
\smallskip

\noindent
Over the past two weeks, both the webapp and server have been under steady
development. On the backend, OAuth scoping has been implemented, meaning that
APIs can only use the endpoints they register for when the user signs up. This
has been used retroactively to harden the security of the /user/ endpoints,
meaning that only staff may perform destructive operations. Additionally, the
/programme/ endpoint has been created, allowing for operations pertaining to
mentorship programmes, and will later form the basis for the multiple
cohorts/classes a programme may have. The cohorts are currently being developed
and tested, in line with the test-driven development policy of the backend
development.

On the front end side, we added functionalities to the login page and register
page, the app is successfully connected to the back end server now, users could
register and login their own account. Also for the register program page, we
modified design and made it more user-friendly, all the data are binded, once
the  programme endpoint in back-end ready user could register the mentor
programme they want. Furthermore, we added a lot of UI components, the app now
looks much better.

\bigskip
\noindent{\Large \bf Meetings To Date}
\smallskip

\noindent
\begin{itemize}
    \item \textbf{Team Meeting (2017/03/01)}: In this meeting, we met with each
        other to debug problems with integrating our backend with the webapp.
        Since we’re following a policy of continuous integration, we aim to have
        our project integrated as early as possible to minimize any issues
        towards the end of the project. Matt demonstrated how to use OAuth2.0 to
        authenticate users on the server, as well as fixing problems with CORS\@.
    \item \textbf{Team Meeting (2017/03/08)}: In this meeting, we met to present
        a progress report to Dr Yun Fu, and we worked together to debug further
        problems with access control. 
\end{itemize}

\bigskip
\noindent{\Large \bf Completed Tasks}
\smallskip

\noindent
Since the previous report these are the tasks that have been completed to date:
\begin{itemize}
    \item Fixed CORS headers so that the client can communicate with the server
    \item Add OAuth scoping to the /user/ endpoints
    \item Add /programme/ endpoints
    \item Create tests for OAuth-protected endpoints
    \item Write unit tests for the Cohort model and serializer
    \item Authentication (register and login account) succeeded.
    \item UI improved.
    \item Register program page finished
\end{itemize}

\bigskip
\noindent{\Large \bf Chris' Progerss}
\smallskip

\noindent
For the last two weeks, I have been working on completing the UI design for the
whole web application. I have been constantly adding different UI components and
also worked on improving user interaction by writing AngularJS directives to
make the UI components interactive. Also I have written controllers for the
profile page which binds user data obtained from the backend so that data of the
currently logged in user can be displayed.

\bigskip
\noindent{\Large \bf Harry's Progress}
\smallskip

\noindent
Last two weeks I continued working on the front-end side. I successfully
connected the front end Auth part to the backend side, include user account
register and login. Also I figured out the data model for register program.
Other small changes include modifying home pages, added search boxes and
functionalities, and research for uploading pictures in the front-end.

\bigskip
\noindent{\Large \bf Matt's Progress}
\smallskip

\noindent
I worked exclusively on the server side again, implementing various model
endpoints as well as ensuring that all endpoints are well tested. Our build
system was given a nice test when I merged two branches into master and deployed
them to the server, and has been successful so far\textemdash~no failed builds have ever
been pushed into the master branch. I also helped to debug issues with the
webapp reaching the server, an issue caused by Chrome’s Cross-Origin Resource
Sharing policies.

\bigskip
\noindent{\Large \bf Plan For The Next Two Weeks}
\smallskip

\noindent
\begin{itemize}
    \item Finish implementing the Cohort model, and implement the Candidate and Mentorship models.
    \item Create endpoints for above models.
    \item Write documentation for the models.
    \item Write helpers to deal with image uploading and storage.
    \item Implement matching endpoint
    \item Add functionalities to profile page and allow users to modify their profile.
\end{itemize}

\hrule

\subsection*{Keeping Up With The Project}

To complement these biweekly reports, you can also view these pages:\\

\noindent
Project site: http://students.cs.ucl.ac.uk/2016/group19/\\

\noindent
Code repository: https://github.com/mbellgb/syseng19-code\\

\noindent
Documentation: https://github.com/mbellgb/syseng19-docs

\end{document}
% vim: textwidth=80
