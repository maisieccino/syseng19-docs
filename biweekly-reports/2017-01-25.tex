% arara: xelatex: { synctex: true, action: nonstopmode, options: "-halt-on-error" }
% arara: clean: { files: [ 2017-01-25.aux, 2017-01-25.bbl, 2017-01-25.bcf, 2017-01-25.blg, 2017-01-25.log, 2017-01-25.out, 2017-01-25.run.xml ]}
\documentclass[11pt]{report}
\usepackage{color}
\usepackage{hyperref}
\hypersetup{colorlinks = true}
\pagenumbering{roman}
\topmargin=0.0in
\oddsidemargin=0.0in
\evensidemargin=0.0in
\textwidth=6.5in
\marginparwidth=0.5in
\headheight=0pt
\headsep=0pt
\textheight=9.0in
\begin{document}

\centerline{{\LARGE \bf Atos Mentor Matching}}

\centerline{\large \bf Biweekly Report 5, 25\textsuperscript{th} of January 2017}
\centerline{Christopher Lau $\bullet$ Harry Chen $\bullet$ Matthew Bell }
\noindent
\line(1,0){470}\\


\noindent {\Large \bf Overview}
\smallskip

\noindent
Over the last two weeks, the team spent most of the team setting up the
development environment. The team also successfully exported the ionic creator
prototype into an ionic project. However, the design is still subject to change
and the user interface would continued to be improved over the course of
development. Also, an UML diagram illustrating the database model was also
created which would act as a reference for implementation of the actual database
model, along with the initial API endpoints being designed and documented using
the online Swagger framework. Finally, there was a short 2 minute elevator pitch
at the 20th of January where the team presented its progress on the project so
far and the future plans of development along with project deliverables.

\bigskip
\noindent{\Large \bf Meetings To Date}
\smallskip

\noindent
\begin{itemize}
    \item \textbf{Team Meeting 1 (2017/01/18)}: This was a short meeting to
        complete the slides for the 2 minute elevator pitch. We summarised our
        progress for term 1 and also completed an system architecture design
        which is included in our elevator pitch. We also established
        deliverables and goals for the end of term 2.
    \item \textbf{Team Meeting 2 (2017/01/25)}: For this meeting, Matt was
        working on completing the UML diagram for the database model while Harry
        and Chris focused on exporting the ionic creator prototype into a ionic
        project and hosting the project locally. We then set further milestones
        for the next two weeks which were mainly to start implementing some of
        the key features of the application.
\end{itemize}

\bigskip
\noindent{\Large \bf Completed Tasks}
\smallskip

\noindent
Since the previous report these are the tasks that have been completed to date:
\begin{itemize}
    \item Exported ionic creator prototype into a working ionic project. Every
        team member is able to host prototype locally.
    \item Completed UML diagram for database model which would act as a
        reference for the actual implementation.
\end{itemize}

\bigskip
\noindent{\Large \bf Chris' Progerss}
\smallskip

\noindent
For this week, I was in charge of summarising the challenges that the team has
faced so far and how we hope to overcome them to include in our elevator pitch.
I have also established some deliverables for the project. Also, I will continue
to learn and carry out research on the Ionic framework so that I can help Harry
for the front-end development.

\bigskip
\noindent{\Large \bf Harry's Progress}
\smallskip

\noindent
In last two weeks I took research of angularjs and Django, also took some
research of database design. I exported our prototype design from ionic creator
and imported it to ionic project then started coding on it. I am now working on
solving routing problem, and connecting front end work with server side (data
base).

\bigskip
\noindent{\Large \bf Matt's Progress}
\smallskip

\noindent
My tasks have been focussed on creating as much useful reference material as
possible to ensure that server development is as painless as it can be. I took
our entity-relation diagram and created a full class diagram, complete with
relations, private/public attributes, and class methods. I have also begun using
this class diagram to design API endpoints using Swagger, an online tool that
allows developers to write API documentation and generate code stubs from it.
This means that when development will be simpler when it comes to writing the
backend itself.

\bigskip
\noindent{\Large \bf Plan For The Next Two Weeks}
\smallskip

\noindent
For the next two weeks, the team plans to start actual implementation of the
features based on the key requirements of the application such as user
authentication, and registration. We will also start implementing the database
model and the API for the backend.

\hrule

\subsection*{Keeping Up With The Project}

To complement these biweekly reports, you can also view these pages:\\

\noindent
Project site: http://students.cs.ucl.ac.uk/2016/group19/\\

\noindent
Code repository: https://github.com/mbellgb/syseng19-code\\

\noindent
Documentation: https://github.com/mbellgb/syseng19-docs

\end{document}
% vim: textwidth=80
