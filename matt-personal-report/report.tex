\documentclass[14pt]{report}
\usepackage{color}
\usepackage{hyperref}
\usepackage[letterpaper,margin=1.5in]{geometry}
\usepackage{fontspec}
\usepackage{titlesec}

\setmainfont{Crimson Text}
\hypersetup{
  colorlinks = true
}

\title{Individual Report}
\date{9th January, 2017}
\author{Matthew Bell\\ Team 19}

\begin{document}
    % \maketitle
    \begin{center}
        \Huge{Individual Report}\\
        \LARGE{Matthew Bell, Team 19}
    \end{center}

    \section*{My Contributions}
    As the technical lead, my main focus for the project is to oversee the design and creation of the proof of concept. Therefore, a lot of my efforts on this project contribute towards the systems design, and I expect my contribution to be greater in the second half of the project, in line with this role.

    I have been assisting with project documentation, mainly by typesetting our reports and other literature, as well as setting up the project website\textemdash creating the basic theme and site skeleton, adding some content, deploying it to the web server.

    I have been deeply involved with the research and design of the main matching algorithm, namely researching the Gale-Shapely algorithm, and adapting it to fit the requirements of our project. I also designed high-level system flow diagrams to demonstrate how a user may interact with the interface, and how that interface would interact with our backend. In doing so, I also provided an overview of the verbs and endpoints our API would be using.

    Security is an essential part of a web-facing application, so I spent some time designing a basic authentication system, which is based upon research of existing systems. We're planning on using JSON Web Tokens (JWTs), which are generated by the server and passed via the user interface and to the user's session. The advantage of JWTs over traditional methods is that they are stateless and don't require any database queries, thus reducing a lot of overhead in our app. This is extremely beneficial since our system will be running on Python, an interpreted language.

    Towards the end of the first term, I began setting up our codebase in a shared repository, and set up project directories and a skeleton web server, so that we can hit the ground running once we return later in January. The advantage of this means that we can also write unit tests for our functions as we create the system, which means that our code will be more reliable. I also plan on setting up a continuous integration server (either self-hosted or using a cloud server like Travis) so that code can be tested automatically. Finally, I have been researching various Python libraries that we can use for certain functions, for instance scheduling system events in the future programmatically. It's not worth us spending more time than necessary reinventing the wheel!

    Throughout the project I have been assisting our team leader by setting up and maintaining certain management tools, such as our shared cloud storage, and our Kanban board. We chose Trello to manage tasks, since it is simple to use and allows tasks to be grouped into lists and labelled.

    \newpage

    \section*{Project Assessment}
    \subsection*{Experimental Phases Worked On}
    As the technical lead, I worked more on the experimental phases that concerned the research and design of our technical prototype. To begin with, I focussed on researching the existing matching algorithms available, and quickly discovered the aforementioned Gale-Shapely algorithm, which I found to fit our problem rather well. Therefore, I would judge the research phase to be extremely successful.

    I also led the research and experimentation of the technology frameworks we are to use for the prototype. I researched the various web server frameworks currently in popular use, and collected as much information on them as possible, including how suitable they are for the project, the advantages and disadvantages of the language, and also how active the development community is. Harry, one of my teammates, researched interface frameworks and also presented his findings. I then made each of us rank the frameworks in order of preference for use, since I did not think that I should be the only one to decide, because we would all be using the chosen system.

    \subsection*{Initial Proof Of Concept}
    The initial proof of concept interface design is an mockup created using Ionic, a web framework that allows developers to write apps for web and mobile with the same code. The accompanying Ionic Designer was used to quickly create an interactive proof of concept to demonstrate how a user would interact with the app. I think that, although there are minor visual bugs such as a few buttons not working, that the overall flow of the system is very good and fits the needs of the users well. However, I think that certain screens could be reordered, such as moving a form where the user can enter their interests from a mentorship sign up page to the initial onboarding process, so that the information only has to be entered once.

    \subsection*{Decision Making}
    I think the team's decision making has been very good so far \textendash although I'm personally not a fan of the chosen frameworks we'll be using, both of the other team members had a strong preference for them. Since I can adapt to the chosen framework, it means that everyone can still be comfortable developing with it, which is the most important thing when choosing technologies to use. I think other decisions, such as how the project is being managed, are also sensible and no one is making decisions without consulting other team members first. We have regular contact, even outside of lab time, by using a group messenger, which helps to faciliate openness.

    \subsection*{Future Of The Project}
    The next steps should be about getting these designs and prototypes translated into a fully-working proof of concept. We have all of the frameworks neccessary \textendash a decentralised cloud-based repository to store our code, a virtual machine to run and test the system in a real-time, production environment, as well as having a clear understanding of the tasks that need to be done and how they'll be split between team members. As soon as classes resume again, the first step will be to have a meeting to catch up and brief each other on how we'll approach development.

    \section*{Team Assessment}

    \subsection*{Christopher Lau}
    \subsubsection*{Primary Role}
    \subsubsection*{Secondary Role}
    \subsubsection*{Strengths And Weaknesses}
    \subsubsection*{Assessment}
    \begin{itemize}
      \item test.
    \end{itemize}

    \subsection*{Harry Chen}
    \subsubsection*{Primary Role}
    \subsubsection*{Secondary Role}
    \subsubsection*{Strengths And Weaknesses}
    \subsubsection*{Assessment}
    \begin{itemize}
      \item test.
    \end{itemize}

    \subsection*{Matthew Bell (Myself)}
    \subsubsection*{Primary Role}
    \subsubsection*{Secondary Role}
    \subsubsection*{Strengths And Weaknesses}
    \subsubsection*{Assessment}
    \begin{itemize}
      \item test.
    \end{itemize}
\end{document}
