\documentclass[14pt]{report}
\usepackage{color}
\usepackage{hyperref}
\usepackage[letterpaper,margin=1.5in]{geometry}
\usepackage{fontspec}
\usepackage{titlesec}

\setmainfont{Crimson Text}
\hypersetup{
  colorlinks = true
}

\title{Individual Report}
\date{9th January, 2017}
\author{Matthew Bell\\ Team 19}

\begin{document}
    \maketitle

    \section*{My Contributions}
    As the technical lead, my main focus for the project is to oversee the design and creation of the proof of concept. Therefore, a lot of my efforts on this project contribute towards the systems design, and I expect my contribution to be greater in the second half of the project, in line with this role.

    I have been assisting with project documentation, mainly by typesetting our reports and other literature, as well as setting up the project website\textemdash creating the basic theme and site skeleton, adding some content, deploying it to the web server.

    I have been deeply involved with the research and design of the main matching algorithm, namely researching the Gale-Shapely algorithm, and adapting it to fit the requirements of our project. I also designed high-level system flow diagrams to demonstrate how a user may interact with the interface, and how that interface would interact with our backend. In doing so, I also provided an overview of the verbs and endpoints our API would be using.

    Security is an essential part of a web-facing application, so I spent some time designing a basic authentication system, which is based upon research of existing systems. We're planning on using JSON Web Tokens (JWTs), which are generated by the server and passed via the user interface and to the user's session. The advantage of JWTs over traditional methods is that they are stateless and don't require any database queries, thus reducing a lot of overhead in our app. This is extremely beneficial since our system will be running on Python, an interpreted language.

    Towards the end of the first term, I began setting up our codebase in a shared repository, and set up project directories and a skeleton web server, so that we can hit the ground running once we return later in January. The advantage of this means that we can also write unit tests for our functions as we create the system, which means that our code will be more reliable. I also plan on setting up a continuous integration server (either self-hosted or using a cloud server like Travis) so that code can be tested automatically. Finally, I have been researching various Python libraries that we can use for certain functions, for instance scheduling system events in the future programmatically. It's not worth us spending more time than necessary reinventing the wheel!

    Throughout the project I have been assisting our team leader by setting up and maintaining certain management tools, such as our shared cloud storage, and our Kanban board. We chose Trello to manage tasks, since it is simple to use and allows tasks to be grouped into lists and labelled.
\end{document}
